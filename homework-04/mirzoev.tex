\documentclass{article}

\usepackage[english,russian]{babel}
\usepackage[T2A]{fontenc}
\usepackage[utf8]{inputenc}
\usepackage{indentfirst}
\usepackage{amsmath}

\title{Комбинаторика\\Домашнее задание 4:\\Рекуррентные соотношения и числа Каталана}
\author{Денис Мирзоев}
\date{}

\begin{document}

\maketitle

\section{}

\subsection{$a_{n+2}=2a_{n_1}-a_n$}

$$a_{n+2}=2a_{n_1}-a_n$$
$$x^2-2x+1=0$$
$$D=4-4=0$$
$$x_{1,2}=1$$
$$a_n=c_1\cdot 1^n+c_2\cdot n\cdot 1^n=c_1+c_2n$$
$$\begin{cases}
    2=c_1\\
    3=c_1+c_2
\end{cases}$$
$$c_2=1$$

Ответ: $a_n=2+n$.

\subsection{$a_{n+2}=2\sqrt{2}a_{n+1}-4a_n$}

$$a_{n+2}=2\sqrt{2}a_{n+1}-4a_n$$
$$x^2=2\sqrt{2}x+4=0$$
$$D=8-16=-8$$
$$x_{1,2}=\frac{2\sqrt{2}\pm i 2\sqrt{2}}{2}=\sqrt{2}\pm i \sqrt{2}$$
$$a_n=c_1(\sqrt{2}+i\sqrt{2})^n+c_2(\sqrt{2}-i\sqrt{2})^n$$
$$2=c_1(\sqrt{2}+i\sqrt{2})^0+c_2(\sqrt{2}-i\sqrt{2})^0$$
$$\begin{cases}
    2=c_1+c_2\Rightarrow c_1=2-c_2\\
    1=c_1(\sqrt{2}+i\sqrt{2})+c_2(\sqrt{2}-i\sqrt{2})
\end{cases}$$
$$1=(2-c_2)(\sqrt{2}+i\sqrt{2})+c_2(\sqrt{2}-i\sqrt{2})$$
$$1=2(\sqrt{2}+i\sqrt{2})-c_2(\sqrt{2}+i\sqrt{2})+c_2(\sqrt{2}-i\sqrt{2})$$
$$1-2\sqrt{2}-i2\sqrt{2}=-c_2 2i\sqrt{2}$$
$$c_2=\frac{1-2\sqrt{2}-i2\sqrt{2}}{2i\sqrt{2}}=i^{-1}(\frac{1}{2\sqrt{2}}-1)-1$$
$$c_1=3-i^{-1}(\frac{1}{2\sqrt{2}}-1)$$
Пусть $t=i^{-1}(\frac{1}{2\sqrt{2}}-1)$. Тогда
$$c_1=3-t$$
$$c_2=t-1$$

Ответ: $a_n=(3-t)(\sqrt{2}+i\sqrt{2})^n+(t-1)(\sqrt{2}-i\sqrt{2})^n$.

\section{}

$$a_{n+5}=-5a_{n+4}+81a_{n+1}+405a_n$$
$$x^5+5x^4-81x-405=0$$
$$(x-3)(x+3)(x+5)(x^2+9)=0$$
$$\text{корни:}\{3,-3,-5,3i,-3i\}$$
Ответ: $a_n=c_1\cdot 3^n+c_2\cdot (-3)^n+c_3\cdot (-5)^n+c_4\cdot(3i)^n+c_5\cdot(-3i)^n$.

\section{}

$$a_{n+2}=5a_{n+1}-6a_n+6\cdot3^n$$
$$x^2-5x+6=0$$
$$x_{1,2}=\{2,3\}$$
Будем искать частное решение вида $cn3^n$.
$$a_n=c_12^n+c_23^n+cn3^n$$
$$c(n+2)3^{n+2}=5c(n+1)3^{n+1}-6cn3^n+6\cdot3^n  | :3^n$$ 
$$c(n+2)9=5c(n+1)33-6cn+6$$
$$9c(n+2)-15c(n+1)+6cn-6=0$$
$$c(9n+18-15n-15+6n)=6$$
$$3c=6$$
$$c=2$$

Ответ: $a_n=c_12^n+c_23^n+2n3^n$.

\section{}

$$a_0=1000$$
$$a_{n+1}=\frac{105}{100}a_n+500$$
$$a_{n+1}=\frac{21}{20}a_n+500$$
$$20a_{n+1}=21a_n+10000$$
$$20c=21c+10000$$
$$c=-10000$$
$$20x-21=0$$
$$x=\frac{21}{20}$$
$$a_n=c_1\Big(\frac{21}{20}\Big)^n-10000$$
$$1000=c_1-10000$$
$$c_1=11000$$

Ответ: $a_n=11000\big(\frac{21}{20}\big)^n-10000$.

\section{}

$$a_{n+1} = a_n + 2n; a_1=2$$
$$x^{n+1} = x^n$$
$$x = 1$$
Будем искать частное решение вида $cn(n-1)$.
$$c(n+1)n=cn(n-1)+2n$$
$$cn(n+1-n+1)=2n$$
$$2nc=2n$$
$$c=1$$
$$a_n=c_1+n(n-1)$$
$$2=a_1=c_1+1(1-1)=c_1$$
$$c_1=2$$

Ответ: $a_n=2+n(n-1)$.

\section{}

$a_n$ - число путей длины n.\\
$b_n$ - число путей длины n заканчивающихся на L.\\
$b_n=a_{n-1}$\\
$c_n$ - число путей длины n заканчивающихся на U.\\
$c_n=a_{n-1}$\\
$d_n$ - число путей длины n заканчивающихся на R.\\
$d_n=a_n-b_n-c_n=a_n-a_{n-1}-a_{n-1}=a_n-2a_{n-1}$\\
$a_n=2b_{n-1}+3(c_{n-1}+d_{n-1})=$\\
$=2a_{n-1}+3(a_{n-2}+(a_{n-1}-2a_{n-2}))=$\\
$=3a_{n-1}-a_{n-2}$\\

Выведем явную формулу.

$a_{n+2}=3a_{n+1}-a_n$

$x^2-3x+1=0$

$x_{1,2}=\frac{3\pm\sqrt{5}}{2}$

$a_n=c_1(\frac{3+\sqrt{5}}{2})^n+c_2(\frac{3-\sqrt{5}}{2})^n$

$a_0=1$\quad$a_1=3$

$
\begin{cases}
    1=c_1+c_2 \Rightarrow c_1=1-c_2\\
    3=c_1\frac{3+\sqrt{5}}{2}+c_2\frac{2-\sqrt{5}}{2} \qquad|\cdot2
\end{cases}
$

$6=(1-c_2)(3+\sqrt{5})+c_2(3-\sqrt{5})$\\
$6=3+\sqrt{5}-c_2(3+\sqrt{5})+c_2(3-\sqrt{5})$\\
$c_2(3+\sqrt{5}-3+\sqrt{5})=-3+\sqrt{5}$\\
$c_2=\frac{-3+\sqrt{5}}{2\sqrt{5}}$\\
$c_1=\frac{2\sqrt{5}+3-\sqrt{5}}{2\sqrt{5}}=\frac{3+\sqrt{5}}{2\sqrt{5}}$

Ответ: $a_n=\frac{3+\sqrt{5}}{2\sqrt{5}}(\frac{3+\sqrt{5}}{2})^n-\frac{3-\sqrt{5}}{2\sqrt{5}}(\frac{3-\sqrt{5}}{2})^n$

\section{}

Рассмотрим правильную скобочную последовательность длины $2n$. Будем идти по
ней слева направо. Встретив открывающую скобку сделаем шаг $(1,0)$, встретив
закрывающую сделаем шаг $(0,1)$. Так как последовательность правильная, то в
итоге мы прийдём в $(n,n)$ и не пересечём $y=x$.

Понятно, что сделав обратную замену отразков на скобки, мы получим исходную
последовательность.

Мы установили биекцию между множеством правильных скобочных последовательностей
длины $2n$ и количеством путей из $(0,0)$ в $(n,n)$ ограниченных указаными
правилам. Число первых описывается числами Каталана, а значит и вторых.

\section{}

Построим рекуррентное выражение для числа $a_n$ расстановок скобок в выражениях
такого вида.

Первая скобка может заключать от $1$ до $n$ множителей(если бы она могла
заключать $n+1$ членов, то число расстановок скобок было бы неограничено). Если
первая скобка заключает $k$ множителей, то число вариантов расстановки скобок
внутри неё равно $a_{k-1}$. Число же вариантов расстановки скобок для оставшихся
множителей равно $a_{n-k}$. То есть общее число расстановок скобок равно

$$a_n=\sum_{k=1}^{n}a_{k-1}a_{n-k}$$

Заменим индекс суммирования на $i=k-1$.

$$a_n=\sum_{k=0}^{n-1}a_ia_{n-1-i}$$

Заметим, что это равно числу Каталана $C_n$. Кроме того $a_1=1=C_1$.

\section{}

Обозначим $a_n$ - число путей Моцкина длины $n$.

Путь Моцкина длины $1$ есть только один: $(1,0)$.

Путей Моцкина длины $2$ есть два: $(1,0) (1,0)$; $(1,1) (1,-1)$.

Найдём количество путей Моцкина длины $n$. Рассмотрим первый шаг. Можно пойти
вправо, вариантов пройти оставшийся путь будет $a_{n-1}$, а можно пойти наверх.
В последнем случае нам рано или поздно придётся сделать шаг вниз. Обозначим за
$k$ число шагов до возврата на ось абсцисс. Число вариантов пройти путь после
шага наверх тогда можно посчитать как $a_ka_{n-2-k}$, где $a_k$ это число
вариантов сделать $k$ шагов без спусков вниз, а $a_{n-2-k}$ - число вариантов
сделать оставшиеся $n-2-k$ шагов. Тогда общее количество путей Моцкина можно
выразить как 

$$a_n=a_{n-1}+\sum_{k=0}^{n-2}a_ka_{n-2-k}\qquad a_0=1\qquad a_1=1$$

\section{}

Для первого шага есть два варианта: вправо или вправо вверх. Если сделали шаг
вправо вверх, то есть $S_{n-1}$ способов пройти остаток пути. Если сделали шаг
вправо, то рано или поздно нам нужно будет сделать шаг вверх, чтобы вернуться
на диагональ. Пусть до возврата на диагональ король сделает $i+1$ шагов($+1$
это шаг вверх). Число способов сделать $i$ шагов до возврата на диагональ:
$S_i$.  Число способов сделать оставшиеся до $(n,n)$ шаги: $S_{n-1-i}$.

$$S_n=S_{n-1}+\sum_{i=0}^{n-1}S_iS_{n-1-i}\qquad S_0=1\qquad S_1=2$$

\end{document}
