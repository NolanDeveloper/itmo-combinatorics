\documentclass{article}

\usepackage[utf8]{inputenc}
\usepackage[T2A]{fontenc}
\usepackage[russian]{babel}
\usepackage{fullpage}
\usepackage{amssymb}
\usepackage{mathtools}

\title{Комбинаторика\\Домашняя работа 3}
\author{Мирзоев Денис}
\date{}

\begin{document}

\maketitle

\section{}

Выберем первую цифру: 5 нечётных и 4 чётных варианта(нельзя ноль).
Все остальные это пять цифр той же чётности, что и первая(по пять
вариантов в любом случае).

Ответ: $(5+4)\cdot5^5$

\section{} 

Каждый студент может выбрать любое подмножество спецкурсов кроме пустого:
$2^4-1$. Студенты выбирают независимо.

Ответ: $(2^4-1)^8$.

\section{} 

Число делится на 3 тогда и только тогда, когда сумма его цифр делится
на 3. Слагаемое, которое делится на 3 не влияет на делимость суммы,
поэтому можно найти количество четырёхзначных чисел, делящихся на 3.
Всего четырёхзначных чисел: $10^4$, делится на 3 каждое третье.

Ответ: $\frac{10^4-1}{3}$.

\section{}

Всего булевых функций от $n$ аргументов: $2^{2^n}$

Не зависят от одной переменной: ${n\choose 1} \cdot 2^{2^{n-1}}$. Выбираем одну
переменную из $n$. Перенумеруем переменные так, чтобы она была первой.
Значения функции в нижней половине таблицы должны повторять значения функции в
верхней. То есть функция определяется верхней половиной значений из таблицы.
В таблице $2^n$ строк. Половина это $\frac{2^n}{2} = 2^{n-1}$.

Не звисят от двух переменных: ${n\choose 2} \cdot 2^{2^{n-2}}$. Выбираем две
переменные из $n$. Перенумеруем переменные так, чтобы эти две были первыми.
Значения функции в нижней половине таблицы должны повторять значения функции в
верхней. Значения функции в каждой второй четверти должны повторять значения
предыдущей. То есть функция определяется верхней четвертью значений из таблицы.
в таблице $2^n$ строк. Четверть это $\frac{2^n}{4} = 2^{2^n-2}$.

Не зависят от $i$ переменных: ${n\choose i} \cdot 2^{2^{n-i}}$.

Зависят от всех переменных: $\text{``все''} - \text{``не зависят от одной''} -
\text{``не зависят от двух''} - \ldots - \text{``не зависят от всех''}$, то
есть $$2^{2^{n-1}}-\sum_{i=1}^{n-1}{2^{2^{n-i}}} $$ $$=
2^{2^{n-1}}-\sum_{i=1}^{n-1}{2^{2^{i}}}$$

\section{}

Выберем грибы для первой связки. Способов это сделать: ${60\choose
15}$.  Осталось 45 грибов, поэтому для второй связки вариантов
${45\choose 15}$.  Для третьей свзки выбираем уже из 30-ти. Вариантов
собрать вторую связку ${30\choose 15}$. Оставшиеся 15 пойдут в
четвёртую связку. Порядок выбора связок важен, поэтому нужно ещё
разделить на число перестановок.

Ответ: $\frac{{60\choose 15}{45\choose 15}{30\choose 15}}{4!}$.

\section{}

$S(n, 3) = \frac{3^n-3(2^n-2)-3}{6}$

Слева записано число неупорядоченых разбиений n-множества на 3 блока.

Это число можно вычислить через упорядоченные разбиения. Упорядоченных
разбиений n-множества на m блоков в $m!$ раз больше, чем
неупорядоченных, поэтому под дробью 6, то есть $3!$.

Сверху записано число упорядоченных разбиений. Первое слагаемое это
число упорядоченных разделений n-множества на 3 блока, то есть все
разбиения n-множества на 3 блока, плюс разделения, в которых
присутствуют пустые блоки. Пустых блоков может быть два или один.

Второе слагаемое это количество упорядоченных разбиений из n-множества
на 3 блока c одним пустым блоком. Пустой блок выбирается тремя
способами(множитель 3), остальные два дужно заполнить n элементами.
Осталось выбрать разбиение n-множества на 2 блока. Делаем это по
похожей схеме: из общего количества разделений вычитаем количество
содержащих пустые блоки. Таких разделений $2^n$ из них 2 имеют пустой
блок(минус 2).

Последнее слагаемое это количество разделений содержащих два пустых
блока. Их три: кладём все объекты в один блок.

\section{}

$n=2$: $0=S(2,0)=\frac{2(2-1)(2-2)(3\cdot 2-5)}{24}=0$

Допустим при числах $\leq n$ формула верна. Докажем, что она верна и при $n+1$.
\begin{align}
    S(n+1, n-1) &= S(n, n-2) + (n-1)\cdot S(n, n-1)\\
    &= \frac{n(n-1)(n-2)(3n-5)}{24} + (n-1)\cdot S(n,n-1)\\
    &= \frac{n(n-1)(n-2)(3n-5)}{24} + (n-1)\cdot {n \choose 2}\\
    &= \frac{n(n-1)(n-2)(3n-5)}{24} + \frac{n(n-1)^2}{2}\\
    &= \frac{n(n-1)(n-2)(3n-5) + 12n(n-1)^2}{24}\\
    &= \frac{n(n-1)((n-2)(3n-5) + 12(n-1))}{24}\\
\end{align}

\begin{align}
    &= \frac{n(n-1)(3n^2-11n+10 + 12n-12))}{24}\\
    &= \frac{n(n-1)(3n^2+n-2)}{24}\\
    &= \frac{n(n-1)(n+1)(3n-2)}{24}\\
    &= \frac{n(n-1)(n+1)(3(n+1)-5)}{24}\\
    &= \frac{(n+1)((n+1)-1)((n+1)-2)(3(n+1)-5)}{24}\\
\end{align}

\section{}

$k^n$ --- число неупорядоченных разделений из $n$ по $k$.

Его можно посчитать как сумму по числу непустых блоков: от $0$ до $n$.
Посчитаем число неупорядоченных разделений, содержащих $i$ непустых блоков.
Выберем неупорядоченное разбиение из $n$ по $i$ $S(n, i)$ способами.  Дадим
каждому блоку порядковый номер в разделении $(k)_i$ способами, первому один из
$k$, второму один из $k-1$ и так далее. Блоки, соответствующие неиспользованных
номерам будут пустыми. То есть мы посчитали число неупорядоченных разделений, содержащих $i$ непустых блоков: $(k)_i\cdot S(n,i)$.

\section{}

Посчитаем число число разбиений $n$-множества. Их можно разделить на два вида:
содержащие одноэлементные блоки и несодержащие.  Первые посчитаны в первом
слагаемом. Установим биекцию между вторыми и разбиениями $n+1$-множества без
одноэлементных блоков, которые посчитаны во втором слагаемом. 

Возьмём разбиение $n$-множества, содержащее хотя бы один одноэлементный блок.
Сольём их в один и добавим в новый блок элемент, которого нет в исходном
множестве. Получили разбиение $(n+1)$-множества без одноэлементных блоков.
Операция обратима: имея разбиение $(n+1)$-множества без одноэлементных блоков, с
зафиксированным элементом, можно удалить его из множества, а все элементы,
содержащиеся в том же блоке раскидать по новым одноэлементным блокам. Тем самым
мы получим исходное разбиение $n$-множества с теми же одноэлементными блоками.

\section{}

Установим биекцию между разбиениями $(n-1)$-множества и разбиениями
$n$-множества, несодержащими последовательно идущих чисел в одном блоке.

Возьмём разбиение $(n-1)$-множества. Добавим во множество ${x_1, \ldots,
x_{n-1}}$ число $x_0$. Сопоставим ему разбиение $n$-множества, несодержащее
последовательно идущих чисел в одном блоке. $x_0$ положим в отдельный блок.
Если $x_i$ и $x_{i+1}$ находятся в одном блоке, перенесём $x_{i+1}$ в блок к
$x_0$. Заметим, что даже в случае $i=1$ $x_0$ и $x_{i+1}$ имеют не
последовательные номера. Будем выполнять эту операцию, пока есть блоки,
содержащие последовательные элементы. В итоге получим разбиение $n$-множества,
несодержащее последовательно идущих элементов в одном блоке. Выполним операцию
в обратном порядке. Рассмотрим блок, содержащий элемент $x_0$. Удалим его из
множества, а все элементы $x_i$ из этого блока перенесём в блоки к элементам
$x_{i-1}$. Получили исходное разбиение.

\end{document}
