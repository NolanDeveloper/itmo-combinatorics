\documentclass{article}

\usepackage[utf8]{inputenc}
\usepackage{fullpage}
\usepackage[russian]{babel}
\usepackage{amssymb}
\usepackage{mathtools}

\def\multiset#1#2{
    \ensuremath{\left(\kern-.3em\left(\genfrac{}{}{0pt}{}{#1}{#2}\right)\kern-.3em\right)}
}

\begin{document}

\section{Мирзоев Денис}

\begin{enumerate}

\item Рассмотрим отдельно n-разрядные числа, где n от 2 до 6. Количество
n-разрядных чисел, удовлетворяющих условию, равно $9^n$. Объяснение: первый
разряд не может быть нулём, значит есть девять вариантов, следующий разряд
может быть любым кроме предыдущего, то есть опять девять вариантов, и так n
раз. Любое одноразрядное число удовлетворяет условию. Их десять.\\
Ответ: $\sum_{n=2}^{6}9^n+10 = 597871$

\item Выложим книги в ряд. Количество способов раскрасить их в три цвета так,
что есть хотя бы одна книга каждого цвета, в этом случае равно количеству
способов расставить две перегородки между ними, не располагая более одной
перегородки между соседними книгами.\\
Ответ: ${12 - 1 \choose 2} = {11 \choose 2} = \frac{11 \cdot 10}{2} = 55$

\item Достаточно выбирать две клетки не лажащие на одной линии. Они будут
задавать все четыре угла этого прямоугольника. Первую ячейку выбираем любую.
Есть $7 \times 7 = 49$ вариантов. Вторая ячейка не должна лежать в том же
столбце или строке. Значит для второй ячейки на $7 + 7 - 1$ вариантов меньше($
- 1$, потому что первую ячейку исключили два раза). Порядок выбора ячеек не
важен, поэтому произведение нужно будет разделить на два.\\
Ответ: $(49 \cdot (49 - 13)) / 2 = (49 \cdot 36) / 2 = 882 $

\item 

\begin{itemize}

\item $\multiset{n + 1}{k}$ - число способов разложить $k$ шаров по $n + 1$ ящику.\\
Рассмотрим случай, когда в ящик с номером n + 1 попало i шаров. Таких случаев
столько же, сколько и способов разложить k - i шаров по оставшимся n ящикам, то
есть $\multiset{n}{k - i}$. В таком случае число способов разложить $k$ шаров
по $n + 1$ ящику можно иначе посчитать как сумму 
$\sum_{i = 0}^{k}\multiset{n}{k-i}$

\item 

\end{itemize}

\item 

$
k = a_1 + \ldots + a_n\\
s = s_1 + \ldots + s_n\\
s \leq k\\
a_i \geq s_i
$

Кажется, что эта задача близка задаче о числе способов разложить k шаров по n
ящикам. Однако ограничение $a_i \geq s_i$ не позволяет провести полную
аналогию. Модифицируем задачу, для проведения аналогии.

$k - s = (a_1 - s_1) + \ldots (a_n - s_n)\\
k - s = a'_1 + \ldots + a'_n$

Количество разложений числа $k - s$ равно искомому количеству разложений и
равно числу способов разложить $k - s$ шаров по n ящикам.
Ответ: $\multiset{n}{k - s}$.

\item Количество способов выбрать одного офицера из трёх: ${3 \choose 1}$.\\
Количество способов выбрать двух сержантов из шести: ${6 \choose 2}$.\\
Количество способов выбрать двадцать рядовых из шестидесяти: ${60 \choose 20}$.\\
Применяем правило суммы.\\
Ответ: ${3 \choose 1} + {6 \choose 2} + {60 \choose 20}$ 

\item 

\begin{itemize}

\item Количество бинарных строк длины $n$, содержащих $k$ единиц?

Оно равно количеству способов выбрать $k$ мест из $n$, то есть ${n \choose k}$.

\item Количество бинарных строк длины $n$, содержащих $k$ единиц, так что
единицы не стоят рядом?

Задача эквивалентна расстановке $k$ перегородок между $n - k$ нулями.
Перегородки не стоят рядом. Их количество равно количеству способов выбрать $k$
мест между нулями из $n - k$, то есть ${n - k \choose k}$.

\end{itemize}

\item Количество k-разрядных чисел из 8 и 9 меньших миллиона: $2^k$.\\
Нам подойдут числа не более 6-разрядов, т.к. с увеличением числа разрядов числа
увеличиваются и наименьшее 7-миразрядное число 8888888 уже больше миллиона.\\
Ответ: $\sum_{k = 1}^{6}{2^k}$.

\item Заполним сначала две одинаковые буквы слова, а потом остальные. Выбрать
позиции для двух одинаковых букв можно ${6 \choose 2}$ способами. Выбрать
букву для этих позиций можно шестью способами. В слове осталось заполнить четыре
буквы. Количество способов сделать это равно $6^4$.\\
Ответ: ${6 \choose 2} \cdot 6 \cdot 6^4$

\item 
Будем рассматривать кости исключительно в такой ориентации: левая половина
содержит не меньше точек, чем правая. Можно расположить их таким образом:

\begin{verbatim}
(0 0)
(1 0) (1 1)
(2 0) (2 1) (2 2)
(3 0) (3 1) (3 2) (3 3)
(4 0) (4 1) (4 2) (4 3) (4 4)
\end{verbatim}

Количество костей с левой половиной содержащей $i$ точек равно $i + 1$. Общее
количество костей тогда можно посчитать как сумму
$$\sum_{i = 0}^{n}{(i + 1)} 
= \frac{1 + (n + 1)}{2} \cdot (n + 1) 
= \frac{(n + 1)(n + 2)}{2} = N$$

Тогда пар костей будет $\frac{N (N - 1)}{2}$.

Рассмотрим теперь количество пар костей из обобщённого домино, которые можно
приложить друг к другу. Если число точек слева и справа различно, то количество
костей, которые можно приставить равно $2n$, в противном случае(дупель) оно
равно $n$. Тогда общее число пар подходящих друг к другу костей можно посчитать
как $\frac{n^2+(N-n)\cdot 2n}{2}$. Первое слагаемое --- количество пар
включающих дупели(их n штук), второе количество пар не включающих дупели(таких
костей $N-n$ штук). Делим на два, потому что посчитали все пары два раза.

\end{enumerate}

\end{document}
